\documentclass{article}
\usepackage[utf8]{inputenc}
\usepackage[margin=1in]{geometry}
\usepackage{amsmath}
\usepackage{hyperref}
\setlength{\parindent}{0em}
\setlength{\parskip}{0.5em}

\title{CTA200H 2024 Assignment 2}
\author{Due by 11:59PM Saturday May 11th}
\date{}

\begin{document}

\maketitle

This assignment will have you practice some basic Python syntax such as functions, for loops and flow control. Write all of your code in a jupyter notebook and save it as `assignment\_2/assignment.ipynb` in your git repo. Make sure you push to Github before the due date.

Note:

For all functions, we requrire that you include a document string, i.e., inside triple quotes, like this:
\break\vskip.05 in
"""\break
This function calculates the sine of x

Parameters:\break
x-----------array-like, the angle in radians

Returns:

sine\_x------array-like, the quantity sin(x)

"""

\section*{Part 1}

Write a python function for the function $f(x) = x^3 - x^2 - 1$. Also, write a function for it's derivative (you will have to work out $df/dx$ yourself), you can call these functions `f' and `df'.

\section*{Part 2}

Write a function `newton(f, df, x0, epsilon=1e-6, max\_iter=30)' which performs a \href{https://en.wikipedia.org/wiki/Newton%27s_method}{Newton Iteration} of the function `f' with derivative `df'.

Newton iteration finds the root ($x_n$ such that $f(x_n) = 0$).

To do this, implement the recursive expression $x_{n+1} = x_n - \frac{f(x_n)}{f'(x_n)}$ using a loop.

The iteration should stop either when `max\_iter' is exceeded or when $|f(x_n)| < \epsilon$.

If the method succeeds, (ie $|f(x_n) < $epsilon), then your function should print `"Found root in <N> iterations"' and should return the value of $x_n$. Otherwise, it should print `"Iteration failed"' and return `None'.

Make sure that your function is documented with \href{https://numpydoc.readthedocs.io/en/latest/format.html}{Numpy style documentation}.

\section*{Part 3}

Try out your function with the function you defined in part 1. You can experiment with setting $x_0$ differently (show at least two examples of $x_0$ in the notebook). Leave epsilon and `max\_iter' as the default values specified in part 2.

Try reducing `epsilon' to 1e-8. Does it still work? If so, how many more iterations does it take to converge.

\section*{How to submit}

Commit the jupyter notebook to your git repo. Push the changes to Github. You do not need to send the link again if you sent it for A1.

\end{document}
